\documentclass{beamer}
\usepackage{amsmath, amsthm, amssymb}
\usepackage{color}
%\setbeamertemplate{items}[ball] 
\usepackage{aecompl,accents}
\usepackage[all]{xy}
\usepackage[normalem]{ulem}

\usepackage{graphicx}
\usepackage{hyperref}
%\usepackage[cache=false]{minted}
 \usepackage{algorithm}
\usepackage{algorithmic}

%%%%%%%%%%%%%%%%%%%%%%%%%%%%%%%%%%%%%%%%%%%%%%%%%%%%%%%%%%%%
\usetheme{Ilmenau}
\usecolortheme{beaver}
\setbeamercolor{block title}{bg=darkred,fg=structure.fg!20!bg!50!bg}
\setbeamercolor{block body}{bg=black!2!red!4!white}
\setbeamercolor{local structure}{fg=darkred}
\setbeamercolor{item projected}{bg=darkred}
\setbeamercolor{titlelike}{parent=structure,bg=red!2!black!5!white}
\setbeamertemplate{title page}[default][rounded=false,shadow=false5]
\setbeamertemplate{enumerate items}[default]
\useinnertheme{rectangles}
\setbeamertemplate{blocks}[default]
\setbeamertemplate{navigation symbols}{}
\usefonttheme[onlymath]{serif}
\usepackage[slovene]{babel}
\usepackage[OT2,T1]{fontenc}
\usepackage[utf8]{inputenc}
%%%%%%%%%%%%%%%%%%%%%%%%%%%%%%%%%%%%%%%%%%%%%%%%%%%%%%%%%%%%%%%%%%%%%%%%%
\newtheorem{trditev}[theorem]{Trditev}
\newtheorem{izrek}[theorem]{Izrek}
\newtheorem{posledica}[theorem]{Posledica}
\newtheorem{definicija}{Definicija}
\newtheorem{domneva}[theorem]{Domneva}
%%%%%%%%%%%%%%%%%%%%%%%%%%%%%
\title[Ničelna prisila]{Ničelna prisila}
\author{Ines Meršak}
\date[13.~april 2018]{Mentor: prof.~dr.~Sandi Klavžar}
%%%%%%%%%%%%%%%%%%%%%%%%%%%%%%%%%%%%%%%%%%%%%%%%%%%%%%%%%%%%%%%%%%%%%%%%%%
\newcommand{\N}{\ensuremath{\mathbb{N}}}
\newcommand{\R}{\ensuremath{\mathbb{R}}}
\newcommand{\F}{\ensuremath{\mathbb{F}}}
\DeclareMathOperator{\rang}{rang}
\DeclareMathOperator{\korang}{korang}
\DeclareMathOperator{\mr}{mr}
\DeclareMathOperator{\M}{M}
\DeclareMathOperator{\supp}{supp}
\newcommand{\squarep}{\mathbin{\square}}


\begin{document}

\frame{\titlepage}

\section{Motivacija}
\begin{frame}[fragile]{Definicija}
    $ G = (V,E)$ končen enostaven neusmerjen graf

    \medskip
    
    \begin{enumerate}
        \item $Z \subset V$ množica črnih vozlišč, $V \setminus Z$ množica belih vozlišč
        \item $\forall u \in Z$, ki ima natanko enega belega soseda $v$, vozlišče $v$ pobarvamo črno
        \item drugo točko ponavljamo, dokler še lahko naredimo kakšno spremembo
    \end{enumerate}
    \begin{definicija}
        \alert{Množica ničelne prisile} je tak $Z \subset V$, da so po koncu zgornjega postopka vsa vozlišča $G$ pobarvana črno.
        $\alert{Z(G)} = \min \{|Z|\colon Z \subset V ,\ Z\text{ je množica ničelne prisile }G \} $
    \end{definicija}
\end{frame}

\begin{frame}{Motivacija}
    $S_n(\F)$ -- simetrične matrike $n\times n$ nad poljem $\F$
    
    \medskip
    
    $A \in S_n(\F) \colon$
    $\mathcal{G}(A)$ je graf z $n$ vozlišči in povezavami $\{\{i,j\}\colon a_{ij} \neq 0, 1 \leq i < j \leq n \} $
    
    \medskip
    \begin{definicija}
        $\mathcal{S}(G) = \{ A \in S_n(\R)\colon \mathcal{G}(A) = G \} $ \\
        $\mr(G) = \min \{ \rang A \colon A \in \mathcal{S}(G) \}$ \alert{minimalni rang} grafa $G$
        $\M(G) = \max \{ \korang A \colon A \in \mathcal{S}(G) \}$ \alert{maksimalni korang} grafa $G$
    \end{definicija}
    
    \[ \mr(G) + \M(G) = |V| \]
\end{frame}

\begin{frame}
    \begin{block}{Problem minimalnega ranga grafa}
        Določiti želimo parameter $\mr(G)$ za nek graf $G$.
    \end{block}
    
    \bigskip
    
    Rešitev tega problema je ekvivalentna rešitvi problema maksimalne večkratnosti lastne vrednosti v družini $\mathcal{S}(G)$.
    
    \bigskip
    
    \begin{block}{Obraten problem lastnih vrednosti grafa}
        Določiti želimo, kakšne so lahko lastne vrednosti matrik iz $\mathcal{S}(G)$.
    \end{block}
\end{frame}

\begin{frame}
    $\supp(x) = \{i \colon x_i \neq 0 \} $ \alert{nosilec} vektorja $x$
    
    \medskip
    
    \begin{trditev}
        $\F$ polje, $A \in \F^{n \times n},\ \korang A > k$ za nek $k \in \N, \ 0 < k < n$.\\
        Za poljubno množico $k$ indeksov $I$ obstaja neničelni vektor $x \in \ker A$, da je $\supp(x) \cap I = \emptyset$.
    \end{trditev}
    
    \begin{trditev}
        $Z$ množica ničelne prisile grafa $G$, $A \in \mathcal{S}(\F,G)$. \\
        Če $x \in \ker A$ in $\supp(x) \cap Z = \emptyset$, je $x = 0$.
    \end{trditev}
    
    \begin{trditev}
        $Z \subseteq V$ množica ničelne prisile grafa $G$. \\ 
        Velja $\M^\F(G) \leq |Z|$ in torej $\M^\F(G) \leq Z(G)$ za poljubno polje $\F$.
    \end{trditev}
\end{frame}

\begin{frame}{$Z(G) = \M(G)$}
    \begin{izrek}
        Za naslednje družine grafov velja $Z(G) = \M(G)$:
        \begin{enumerate}
            \item vsi grafi $G$ z $|G| \leq 6$
            \item $P_n, C_n, K_n$
            \item drevesa
        \end{enumerate}
    \end{izrek}
\end{frame}

\section{Rezultati za tipične družine grafov}
\begin{frame}{Karakterizacija grafov z ekstremnimi $Z(G)$}
    \begin{trditev}
        \[ Z(G) = 1 \iff G = P_n \text{ za } n \geq 1 \]
    \end{trditev}
    
    \bigskip
    
    \begin{trditev}
        Naj bo $G$ povezan graf z $|G| \geq 2$. Potem velja
        \[ Z(G) = |G|-1 \iff G = K_{|G|} \]
    \end{trditev}
\end{frame}

\begin{frame}
    \[ Z(C_n) = 2 \text{ za } n \geq 3 \]
    
    \medskip
    
    \[ Z(S_n) = n-2 \text{ za } n \geq 4 \]
\end{frame}

\section{Nekaj ocen}
\begin{frame}{Kartezični produkt}
    \begin{definicija}
        \alert{Kartezični produkt} grafov $G \squarep H$ je graf, za katerega velja:
        \begin{enumerate}
            \item množica vozlišč je $V(G) \times V(H)$,
            \item vozlišči $(u,u')$ in $(v,v')$  sta sosednji $\iff$
            \begin{itemize}
                \item $u = v$ in $u' \sim v'$ v $H$ ali
                \item $u' = v'$ in $u \sim v$ v $G$.
            \end{itemize}
        \end{enumerate}
    \end{definicija}
    
    \begin{trditev}
        \[ Z(G \squarep H) \leq \min \{Z(G)\cdot|H|, Z(H) \cdot |G| \} \]
    \end{trditev}
    \vspace{-3mm}
    \[ Q_n = Q_{n-1} \squarep K_2 \implies Z(Q_{n}) \leq 2^{n-1} \]
\end{frame}

\begin{frame}{Zgornja meja}
    \begin{izrek}
        Naj bo $G$ graf z $n$ vozlišči, največjo stopnjo vozlišča označimo z $\Delta$ in privzamemo, da je najmanjša stopnja vozlišča vsaj 1.
        \begin{enumerate}
            \item $Z(G) \leq \frac{\Delta}{\Delta + 1} n$
            \item Če je $G$ povezan in velja $\Delta \geq 2$, potem $Z(G) \leq 
            \frac{(\Delta - 2)n + 2}{\Delta - 1}$
        \end{enumerate}
    \end{izrek}
\end{frame}

%\section{Število korakov (propagation time)}
%\begin{frame}{Definicija}
%    
%\end{frame}

\end{document}
